\section{Source code structure and conventions}

\subsection{Approach to portability}

Although Bmboot currently supports only one hardware platform, it is architected with a degree of portability in mind. The approach is illustrated by the implementation of the interrupt controller driver in the monitor.

The interface of the driver is defined in a file named\\
\texttt{platform\_interrupt\_controller.hpp}, \\
which is placed in the common part of the source tree. On the other hand, the path to the ZynqMP-specific implementation is\\
\texttt{platform/zynqmp/executor/monitor/interrupt\_controller.cpp}. \\
If another platform were to be supported, it would need to provide its own implementation of the interface, but reuse the same header.

The platform-agnostic part of the source code still makes strong assumptions about the ARMv8-A architecture. Porting Bmboot to another CPU architecture would require a thorough refactoring.

The author suggests referring to the Porting Guide of the Trusted Firmware-A project \cite{tfa-porting-guide} for an example of a more complete and well-thought-out approach.

\subsection{Namespaces}

\begin{itemize}
    \item \texttt{bmboot}
    \item \texttt{bmboot::internal}
    \item \texttt{bmboot::platform}
\end{itemize}

\subsection{Directories}

\begin{itemize}
    \item header files starting with \texttt{platform\_} declare interfaces (functions and structures) to be implemented by platform-specific code
\end{itemize}
